%%%%%%%%%%%%%%%%%%%%%%%%%%%%%%%%%%%%%%%%%
% Wenneker Article
% LaTeX Template
% Version 2.0 (28/2/17)
%
% This template was downloaded from:
% http://www.LaTeXTemplates.com
%
% Authors:
% Vel (vel@LaTeXTemplates.com)
% Frits Wenneker
%
% License:
% CC BY-NC-SA 3.0 (http://creativecommons.org/licenses/by-nc-sa/3.0/)
%
%%%%%%%%%%%%%%%%%%%%%%%%%%%%%%%%%%%%%%%%%

%----------------------------------------------------------------------------------------
%	PACKAGES AND OTHER DOCUMENT CONFIGURATIONS
%----------------------------------------------------------------------------------------


%\documentclass[10pt, a4paper, twocolumn]{article}
\documentclass[10pt, a4paper]{article}% 10pt font size (11 and 12 also possible), A4 paper (letterpaper for US letter) and two column layout (remove for one column)

\input{structure.tex} % Specifies the document structure and loads requires packages
\usepackage[sc]{mathpazo} % Use the Palatino font
\usepackage[T1]{fontenc} % Use 8-bit encoding that has 256 glyphs
\linespread{1.05} % Line spacing - Palatino needs more space between lines
%\usepackage{microtype} % Slightly tweak font spacing for aesthetics

\usepackage[twoside,width=16cm,height=24cm,left=3cm]{geometry}
%\usepackage[hmarginratio=1:1,top=20mm,width=19cm,height=23cm,columnsep=15pt]{geometry} % Document margins
\usepackage{multicol} % Used for the two-column layout of the document
\usepackage[hang, small,labelfont=bf,up,textfont=it,up]{caption} % Custom captions under/above floats in tables or figures
\usepackage{booktabs} % Horizontal rules in tables
\usepackage{float} % Required for tables and figures in the multi-column environment - they need to be placed in specific locations with the [H] (e.g. \begin{table}[H])
\usepackage{hyperref} % For hyperlinks in the PDF

%----------- Agregados para el caso de ustedes -------------------------------
\usepackage[spanish]{babel}% idioma castellano
\usepackage[utf8]{inputenc}% esto es para poder poner los tildes directamente. Puede que cambie de versión a versión de sistema operativos (más información en http://www.aq.upm.es/Departamentos/Fisica/agmartin/webpublico/latex/FAQ-CervanTeX/FAQ-CervanTeX-6.html )
\usepackage{graphicx} % para insertar figuras
\usepackage{subfigure} % para insertar figuras dentro de figuras
\usepackage{times} % plataforma
\usepackage{amsmath} % --para ecuaciones y algunos símbolos 
% ---------------------- -----------------------------------------------------

\usepackage{lettrine} % The lettrine is the first enlarged letter at the beginning of the text
%\usepackage{paralist} % Used for the compactitem environment which makes bullet points with less space between them


\usepackage{abstract} % Allows abstract customization
\renewcommand{\abstractnamefont}{\normalfont\bfseries} % Set the "Abstract" text to bold
\renewcommand{\abstracttextfont}{\normalfont\itshape} % Set the abstract itself to small italic text
\addto\captionsspanish{ % Modifica algunos nombres cambiandolos por los definidos a continuacion
        \def\contentsname{\'Indice}%
        \def\bibname{Referencias}%
        \def\tablename{Tabla}%
        \def\abstractname{Resumen}
        }
\usepackage[usenames,dvipsnames,svgnames,table]{xcolor}
%\usepackage{natbib}
%\usepackage[usenames]{color}
\usepackage{graphicx}
\usepackage[spanish]{babel}
\usepackage{amsmath}
\usepackage{float}
\usepackage{dsfont}
\usepackage{textcomp}
\usepackage{soul}
\usepackage{fancyhdr}
\usepackage{titlesec} % Allows customization of titles
\usepackage{fancyhdr} % Headers and footers
\usepackage[spanish]{babel}
\usepackage{amsmath}
%\usepackage{hyperref}


\pagestyle{fancy} % All pages have headers and footers
 \fancyhead{} % Blank out the default header
 \fancyfoot{} % Blank out the default footer
\fancyhead[C]{Laboratorio  $\bullet$ $\today$ } % Custom header text
\fancyfoot[RO,LE]{\thepage} % Custom footer text


 %incluye los paquetes usados mas comunes 

%----------------------------------------------------------------------------------------
%	INFORMACION DEL ARTICULO
%----------------------------------------------------------------------------------------
\title{\centering{Estudio y caracterización de modos transversales electromagnéticos y cavidades de oscilación de un láser Nd:YAG}} % Titulo del Informe

\author{
	\authorstyle{Lucía Evangelista Gallo \textsuperscript{1,1}}
	\authorstyle{Leandro Ariel Pezzente\textsuperscript{1,1}} % Authors
	\newline\newline % Space before institutions
	\textsuperscript{1}\institution{Facultad de Ciencias Exactas y Naturales}\\ % Institution 1
	\textsuperscript{1}\institution{Universidad Nacional de Buenos Aires, Buenos Aires, Argentina}\\ % Institution 1
	\keywordname{Laser --- Nd:YAG --- Modos Transversales --- Cavidades de Oscilación } % Keywords - if you don't want any simply remove all the text between the curly brackets
}

\date{} % Add a date here if you would like one to appear underneath the title block, use \today for the current date, leave empty for no date

%----------------------------------------------------------------------------------------

\begin{document}
\maketitle % Print the title
\thispagestyle{fancy} % All pages have headers and footers
%\thispagestyle{firstpage} % Apply the page style for the first page (no headers and footers)

%----------------------------------------------------------------------------------------
%	ABSTRACT
%----------------------------------------------------------------------------------------

\begin{abstract}
%\lettrineabstract{ 
En este experimento se busca estudiar el comportamiento de la cavidad resonante de un láser de Nd:YAG. Se procede primero a determinar las condiciones de alineación bajo las cuales un espejo dieléctrico de 98\% permite la aparición de modos resonantes en una cavidad lineal , así como también se estudian las características de la potencia óptica emitida por el láser de Nd:YAG tanto en cavidades lineales como en cavidades en V. Finalmente, se analizan los distribución espacial de los diferentes modos transversales electromagnéticos emitidos por el laser  
%}
\end{abstract}
\begin{multicols}{2} % Two-column layout throughout the main article text
%----------------------------------------------------------------------------------------
%	ARTICLE CONTENTS
%----------------------------------------------------------------------------------------
\tableofcontents % Print the contents section

\section{Introducción}
% \addcontentsline{toc}{section}{Introduccion} % Adds this section to the table of contents

\subsection{Estructura y composición de un laser}

El láser es un dispositivo que permite la amplificación de radiación electromagnética mediante un proceso físico conocido como emisión estimulada o inducida. Este proceso permite la amplificación de señales lumínicas generadas por otros medios. El mecanismo que genera dichas señales lumínicas se denominada mecanismo de bombeo, y los procesos para generarlo son diversos y pueden ir desde descargas eléctricas en medios gaseoso hasta el uso de otros laseres en la excitación fluorescente de medios orgánicos como colorantes. Esto es necesario para excitar las transiciones electrónicas en el medio en el que se produce la emisión estimulada, denominado medio activo,debido a que se debe entregar energía al medio activo de manera constante para que pueda mantenerse la emisión estimulada. Por ultimo, se necesita de un mecanismo de realimentación a fin de inducir las transiciones electrónicas que producen la amplificación de la radiación electromagnética. El mecanismo de retroalimentación se logra mediante una cavidad resonante que usualmente consiste en dos o mas espejos de alta reflectividad alineados de tal manera de se favorezca la la reinyeccion de la luz emitida en el sentido del eje de la cavidad por el medio activo excitado al mismo medio. Los espejos entonces actúan como un multiplicador de la longitud de onda del material. El mecanismo de retroalimentación es indispensable porque para que la el láser emita, el proceso de bombeo no debe solamente excitar el medio activo, sino que que también debe lograr la condición de población de inversión en la cual hay mas electrones del medio activo en estados excitados de alta energía que en estados de baja energía.
Otro aspecto importante por el cual es crucial la correcta alineación de los espejos en la cavidad, es que , puesto que como el resonador debe sustentar ondas estacionarias de luz, es decir, se debe mantener una configuración estable del campo de radiación.
\newline
Las características mas distintivas de los laseres son que la luz emitida por estos es cuasi-monocromática, coherente y altamente colimada.

\subsection{Condiciones de estabilidad de una cavidad láser }
Se define un resonador óptico o cavidad como el arreglo de componentes ópticos que permite que un haz de luz circule dentro del láser. Para un resonador óptico dado en un láser se tienen modos de oscilación que dependen son o bien dependientes de la frecuencia del haz o bien dependientes de la distribución espacial del haz. Estos últimos modos se denominan modos transversales electromagnéticos, y son una consecuencia del hecho de el haz de luz no es una onda plana sino que posee una distribución espacial finita. Por lo general, un haz láser emitido por una cavidad resonante es una combinación de varios modos de oscilación. Esto es importante al analizar las perdidas por difracción las cuales son mas altas para modos transversales de mayor orden y afectan mas seriamente a la calidad del haz. Además un resonador óptico puede ser estable o inestable , aquí el termino estabilidad esencialmente significa que cualquier rayo de luz inyectado en el sistema inicialmente con algún ángulo y apartamiento de la posición transversal respecto del eje óptico se mantendrá dentro del sistema durante muchos cada viajes cerrados. Para los resonadores inestables, en cambio, hay rayos que pueden exhibir un incremento ilimitado de su posición transversal, eventualmente divergiendo y abandonando el sistema óptico. La estabilidad de un resonador depende de la disposición particular de los componentes de la cavidad, especialmente la curvatura de las superficies reflectantes, la reflectividad de las superficies y la distancia entre los componentes. Si bien la mayoría de los laseres están basados en resonadores estables, los resonadores inestables también presentan ciertas ventajas, particularmente en aquellos con una alta eficiencia y potencia de salida. Sin embargo, los resonadores inestables tienden a tener distribuciones mas complicadas de modos transversales electromagnéticos.
\newline
\textcolor{DarkBlue}{Placeholder para grafico de cavidades estables e inestables}
\newline
Los tipos de cavidades estables frecuentemente utilizadas son la cavidad conformada por dos espejos separados a una distancia d. El caso de dos espejos planos paralelos se conoce como resonador de Fabry-Perot y el caso con dos espejos esféricos de diferentes radios de curvatura se tienen resonadores concéntricos y los resonadores confocales. Para el caso de resonadores de Fabry-Perot la condición de estabilidad se obtiene de imponer que la distancia d entre los espejos, esto es, el largo de la cavidad, sea a primera aproximación de los modos, igual a un numero entero de medias longitudes de onda, ${ d = n \frac{\lambda}{2} }$ donde ${\lambda}$ es la longitud de onda del haz y n es un numero entero.Para el caso de dos espejos esféricos, es de particular interés el resonador de tipo confocal. Este resonador consiste de dos espejos de diferente radio de curvatura R1 y R2 separados a una distancia focales de ambos espejos caigan en el mismo punto. Esta configuración tiene la ventaja es independiente de la dirección en la que viajen los haces de luz. Se tendrá para este caso una cavidad resonante estable si se cumple la condición : 

\begin{equation}
    0 \leq g_1 \cdot g_2 \leq 1
\end{equation}

donde se ha definido ${g_{1,2} = 1 - \frac{d}{R_{1,2}}}$ considerando que los espejos convergente tienen radio de curvatura positivo.

\section{Desarrollo experimental}
La primera parte del experimento consistió en armar una cavidad estable y medir cómo afectaba la variación de corriente y de distancia a la potencia de salida del láser y del diodo de bombeo. Para configurar la cavidad, se utilizó un láser auxiliar de He--Ne (\textcolor{cyan}{marca, modelo})({\textcolor{DarkBlue}{Melles Griot 05-LHR-633? Potencia Maxima 5 mW a 632.8 nm Laser tipo Clase IIIB}}), dos espejos de plata (\textcolor{cyan}{especificación, reflejaban 100 \%? }) ({\textcolor{DarkBlue} {La guia dice 99.99 \%}}) y un espejo plano con 98\% de reflectividad (el espejo que \textcolor{cyan}{cierra} la cavidad). Inicialmente se colocaron los espejos de Ag como muestra la Fig. \ref{cavestable}; el primero se encontraba a una distancia horizontal de $(148.5 \pm 0.1)$cm del láser de He--Ne, el segundo a $(169.5 \pm 0.1)$cm del medio óptico (también en la misma línea) y entre ambos espejos había una distancia diagonal de $(155.3 \pm 0.1)$cm. Los espejos estaban montados en posicionadores angulares que habilitaban todos los grados de libertad necesarios para alinearlos. Se comenzó alineando las reflexiones en ambos espejos. Una vez logrado, se colocó el espejo plano a una distancia de $(4.9 \pm 0.1)$cm de la barrita YAG y se repitió el proceso de alineación. Luego, se alimentó el diodo láser con una corriente de 2.3A aproximadamente y se utilizó una tarjeta infrarroja (la longitud de onda central del láser, indicada por la hoja de datos, es de 1064nm) para verificar que el láser estuviera efectivamente funcionando. 

Con esta cavidad estable ya definida, se tomaron las siguientes mediciones:
\begin{itemize}
    \item Potencia de salida del láser en función de la corriente, para valores entre 0.5A y 2.35A. El barrido se hizo cada 5A entre 2A y 2.39A y cada 10A entre 0.5A y 2A. Distancia fija de $(11.5 \pm 0.1)$cm. 
    \item Potencia de salida del láser a distancias de $(11.5 \pm 0.1)$cm, $(19.0 \pm 0.1)$cm, $(25.0 \pm 0.1)$cm, $(29.1 \pm 0.1)$cm, $(35.5 \pm 0.1)$cm, $(39.7 \pm 0.1)$cm y $(44.2 \pm 0.1)$cm. Corriente fija de 2.39A. 
\end{itemize}
Luego se removió el espejo plano (es decir, se desarmó el láser) y se midió potencia de bombeo del diodo láser en función de la corriente y en función de la distancia manteniendo los parámetros anteriores. \textcolor{cyan}{Falta agregar lo último de diodo láser, lo que Nico dijo que habíamos medido mal.}({\textcolor{DarkBlue}{Que debimos haber medido la potencia del diodo laser en el foco donde esta la barra de Nd-Yag, es decir, retirando la barrita y midiendo solo el laser del diodo y que lo que medimos fue "la potencia a la salida del cristal, a una distancia fija" y que hay que tener en cuenta que esto tiene incorporado un espejo}})


%------------------------------------------------


%------------------------------------------------

\subsection{Subsection}

\begin{enumerate}
	\item First numbered item in a list
	\item Second numbered item in a list
	\item Third numbered item in a list
\end{enumerate}



\begin{table}
	\caption{Example table}
	\centering
	\begin{tabular}{llr}
		\toprule
		\multicolumn{2}{c}{Name} \\
		\cmidrule(r){1-2}
		First Name & Last Name & Grade \\
		\midrule
		John & Doe & $7.5$ \\
		Richard & Miles & $5$ \\
		\bottomrule
	\end{tabular}
\end{table}

%------------------------------------------------

\section{Resultados del Experimento}
% \addcontentsline{toc}{section}{Resultados del Experimento} % Adds this section to the table of contents

Los resultados obtenidos en relación a la potencia tanto del láser como de bombeo fueron los esperados (\textcolor{cyan}{Estoy suponiendo que vamos a rehacer los de bombeo y van a dar lo que esperamos. Sino cambio esta frase})({\textcolor{DarkBlue}{Si volvemos a hacer la medicion y no da, me uno a la compania del anillo y me voy con Frodo hasta la Montala del Destino a arrojar el anillo unico}}). En la Fig. \ref{laservscorr} se ve que la potencia óptica del láser es insignificante hasta que la corriente de bombeo alcanza 1.9A. A partir de ese umbral, la potencia aumenta de forma lineal (\textcolor{cyan}{R$^2$ = ?, $\chi^2$ = ?}) (\textcolor{DarkBlue}{R$^2$ = 0,99192, $\chi^2$ = 0,02477}), con pendiente \textcolor{cyan}{[insertar pendiente]} (\textcolor{DarkBlue}{ pendiente = ${ 11,89 \pm 0,38}$ mW/A , offset ${ -22,81 \pm 0,81 mW/A }$}), lo cual implica una ganancia de [\textcolor{cyan}{algo, si es que vale esta afirmación}]. \textcolor{cyan}{Resultaría interesante estudiar, en alguna experiencia futura, hasta qué valor de corriente se mantiene esta relación lineal}. 

La variación de la potencia óptica con la distancia (Fig. \ref{laservsdist}), a priori, parecen indicar que esta disminuye a medida que se aleja el sensor del láser. Sin embargo, teniendo en cuenta las características de estos dispositivos (\textcolor{cyan}{insertar características})(\textcolor{DarkBlue}{ Medidor de Potencia Optica Thorlabs modelo S302C , linealidad ${\pm 1\%}$ , incerteza de calibracion ${ \pm 3\% @ 1064 nm }$ }), se descarta esta suposición. Se cree que el porqué del comportamiento radica en la dirección del haz; es posible que este estuviera ligeramente desviado con respecto a la línea que une el láser con el sensor (\textcolor{DarkBlue}{El area activa del sensor tiene un diametro de 12 mm , a medida que te alejas aumenta el brazo de palanca por lo que una pequeña desviacion de la direcccion del haz puede tranquilamente salirse del area, ademas esta el angulo de la cabeza del detector respecto al eje del vastago }). Además, a pesar de haber utilizado un obturador, no se debe descartar que parte del haz del diodo de bombeo haya podido filtrarse y su potencia ser medida junto con la del láser. Esto podría explicar el comportamiento de los últimos tres puntos; de haber sido únicamente un problema de dirección del haz, los datos deberían haberse ubicado todos sobre una recta decreciente (como los primeros tres) hasta alcanzar una distancia en que la potencia fuera nula, pues el haz ya no estaría incidiendo sobre el sensor. 

\begin{figure}[H]
    \centering
    \includegraphics[scale=0.3]{Graficos/potlaservscorr.png}
    \caption{Potencia óptica de láser en función de la corriente de bombeo. La potencia aumenta linealmente a partir de 1.9A.}
    \label{laservscorr}
\end{figure}

\begin{figure}[H]
    \centering
    \includegraphics[scale=0.3]{Graficos/potlaservsdist.png}
    \caption{Potencia óptica de láser en función de la distancia para una corriente de bombeo constante de 2.39A.}
    \label{laservsdist}
\end{figure}

\section{Análisis}
% \addcontentsline{toc}{section}{Analisis y Conclusiones} % Adds this section to the table of contents

\begin{figure}
	\includegraphics[width=\linewidth]{bear.jpg} % Figure image
	\caption{A majestic grizzly bear} % Figure caption
	\label{bear} % Label for referencing with \ref{bear}
\end{figure}

\section{Conclusiones}
% \addcontentsline{toc}{section}{Conclusiones} % Adds this section to the table of contents

%----------------------------------------------------------------------------------------
%	BIBLIOGRAPHY
%----------------------------------------------------------------------------------------

\printbibliography[title={Bibliography}] % Print the bibliography, section title in curly brackets

%----------------------------------------------------------------------------------------
\end{multicols}
\end{document}
