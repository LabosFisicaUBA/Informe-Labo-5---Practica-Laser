%%%%%%%%%%%%%%%%%%%%%%%%%%%%%%%%%%%%%%%%%
% Wenneker Article
% LaTeX Template
% Version 2.0 (28/2/17)
%
% This template was downloaded from:
% http://www.LaTeXTemplates.com
%
% Authors:
% Vel (vel@LaTeXTemplates.com)
% Frits Wenneker
%
% License:
% CC BY-NC-SA 3.0 (http://creativecommons.org/licenses/by-nc-sa/3.0/)
%
%%%%%%%%%%%%%%%%%%%%%%%%%%%%%%%%%%%%%%%%%

%----------------------------------------------------------------------------------------
%	PACKAGES AND OTHER DOCUMENT CONFIGURATIONS
%----------------------------------------------------------------------------------------


%\documentclass[10pt, a4paper, twocolumn]{article}
\documentclass[10pt, a4paper]{article}% 10pt font size (11 and 12 also possible), A4 paper (letterpaper for US letter) and two column layout (remove for one column)

\input{structure.tex} % Specifies the document structure and loads requires packages
\usepackage[sc]{mathpazo} % Use the Palatino font
\usepackage[T1]{fontenc} % Use 8-bit encoding that has 256 glyphs
\linespread{1.05} % Line spacing - Palatino needs more space between lines
%\usepackage{microtype} % Slightly tweak font spacing for aesthetics

\usepackage[twoside,width=16cm,height=24cm,left=3cm]{geometry}
%\usepackage[hmarginratio=1:1,top=20mm,width=19cm,height=23cm,columnsep=15pt]{geometry} % Document margins
\usepackage{multicol} % Used for the two-column layout of the document
\usepackage[hang, small,labelfont=bf,up,textfont=it,up]{caption} % Custom captions under/above floats in tables or figures
\usepackage{booktabs} % Horizontal rules in tables
\usepackage{float} % Required for tables and figures in the multi-column environment - they need to be placed in specific locations with the [H] (e.g. \begin{table}[H])
\usepackage{hyperref} % For hyperlinks in the PDF

%----------- Agregados para el caso de ustedes -------------------------------
\usepackage[spanish]{babel}% idioma castellano
\usepackage[utf8]{inputenc}% esto es para poder poner los tildes directamente. Puede que cambie de versión a versión de sistema operativos (más información en http://www.aq.upm.es/Departamentos/Fisica/agmartin/webpublico/latex/FAQ-CervanTeX/FAQ-CervanTeX-6.html )
\usepackage{graphicx} % para insertar figuras
\usepackage{subfigure} % para insertar figuras dentro de figuras
\usepackage{times} % plataforma
\usepackage{amsmath} % --para ecuaciones y algunos símbolos 
% ---------------------- -----------------------------------------------------

\usepackage{lettrine} % The lettrine is the first enlarged letter at the beginning of the text
%\usepackage{paralist} % Used for the compactitem environment which makes bullet points with less space between them


\usepackage{abstract} % Allows abstract customization
\renewcommand{\abstractnamefont}{\normalfont\bfseries} % Set the "Abstract" text to bold
\renewcommand{\abstracttextfont}{\normalfont\itshape} % Set the abstract itself to small italic text
\addto\captionsspanish{ % Modifica algunos nombres cambiandolos por los definidos a continuacion
        \def\contentsname{\'Indice}%
        \def\bibname{Referencias}%
        \def\tablename{Tabla}%
        \def\abstractname{Resumen}
        }
\usepackage[usenames,dvipsnames,svgnames,table]{xcolor}
%\usepackage{natbib}
%\usepackage[usenames]{color}
\usepackage{graphicx}
\usepackage[spanish]{babel}
\usepackage{amsmath}
\usepackage{float}
\usepackage{dsfont}
\usepackage{textcomp}
\usepackage{soul}
\usepackage{fancyhdr}
\usepackage{titlesec} % Allows customization of titles
\usepackage{fancyhdr} % Headers and footers
\usepackage[spanish]{babel}
\usepackage{amsmath}
%\usepackage{hyperref}


\pagestyle{fancy} % All pages have headers and footers
 \fancyhead{} % Blank out the default header
 \fancyfoot{} % Blank out the default footer
\fancyhead[C]{Laboratorio  $\bullet$ $\today$ } % Custom header text
\fancyfoot[RO,LE]{\thepage} % Custom footer text


 %incluye los paquetes usados mas comunes 

%----------------------------------------------------------------------------------------
%	INFORMACION DEL ARTICULO
%----------------------------------------------------------------------------------------
\title{\centering{Estudio y caracterización de modos transversales electromagnéticos y cavidades de oscilación de un láser Nd:YAG}} % Titulo del Informe

\author{
	\authorstyle{Lucía Evangelista Gallo \textsuperscript{1,1}}
	\authorstyle{Leandro Ariel Pezzente\textsuperscript{1,1}} % Authors
	\newline\newline % Space before institutions
	\textsuperscript{1}\institution{Facultad de Ciencias Exactas y Naturales}\\ % Institution 1
	\textsuperscript{1}\institution{Universidad Nacional de Buenos Aires, Buenos Aires, Argentina}\\ % Institution 1
	\keywordname{Laser --- Nd:YAG --- Modos Transversales --- Cavidades de Oscilación } % Keywords - if you don't want any simply remove all the text between the curly brackets
}

\date{} % Add a date here if you would like one to appear underneath the title block, use \today for the current date, leave empty for no date

%----------------------------------------------------------------------------------------

\begin{document}
\maketitle % Print the title
\thispagestyle{fancy} % All pages have headers and footers
%\thispagestyle{firstpage} % Apply the page style for the first page (no headers and footers)

%----------------------------------------------------------------------------------------
%	ABSTRACT
%----------------------------------------------------------------------------------------

\begin{abstract}
%\lettrineabstract{ 
En este experimento se busca estudiar el comportamiento de la cavidad resonante de un láser de Nd:YAG. Se procede primero a determinar las condiciones de alineación bajo las cuales un espejo dieléctrico de 98\% permite la aparición de modos resonantes en una cavidad lineal , así como también se estudian las características de la potencia óptica emitida por el láser de Nd:YAG tanto en cavidades lineales como en cavidades en V. Finalmente, se analizan los distribución espacial de los diferentes modos transversales electromagnéticos emitidos por el laser  
%}
\end{abstract}
\begin{multicols}{2} % Two-column layout throughout the main article text
%----------------------------------------------------------------------------------------
%	ARTICLE CONTENTS
%----------------------------------------------------------------------------------------
%\tableofcontents % Print the contents section

\section*{Introducción}
% \addcontentsline{toc}{section}{Introduccion} % Adds this section to the table of contents

Un láser es un dispositivo que permite la amplificación de radiación electromagnética mediante un proceso físico conocido como emisión estimulada o inducida. Estos instrumentos están compuestos por un mecanismo de bombeo, un medio amplificador y un medio de realimentación. En los experimentos realizados, se utilizó un diodo láser como mecanismo de bombeo y una barra de Nd:YAG como medio amplificador. Para el caso de luz, una cavidad resonante (el medio de realimentación) consiste en un arreglo de espejos tal que la luz recorre el mismo camino óptico varias veces. Uno de esos espejos debe tener reflectividad menor al 100\% para permitir la salida de un haz. Se logran distintas cavidades modificando las distancias y los radios de curvatura de los espejos. Dado que se buscaba un haz continuo, se diseñó una cavidad estable, es decir, un recinto donde, dados dos espejos de radios R$_1$ y R$_2$ separados por una distancia $d$, se cumple la siguiente relación:
\begin{equation}
    0 \leq g_1 g_2 \leq 1,
\end{equation}
con $g_i = 1 - d/R_i$. Es importante tener en cuenta que, para que el dispositivo emita luz, las dimensiones de la cavidad deben ser (un número entero de veces) proporcionales a la longitud de onda de emisión, que en este caso era 1064nm.

Las características más distintivas de los láseres son que la luz emitida por éstos es cuasi-monocromática (ancho de línea espectral pequeño), coherente (diferentes puntos del campo eléctrico oscilan con la misma diferencia de fase), y altamente colimada (baja divergencia del haz).


Uno de los objetivos de este trabajo fue visualizar distintos modos TEM$_{pq}$ (modo transverso electromagnético de orden pq). Dadas las dimensiones de la barra YAG y de los espejos (que no tienen radio infinito), no se pueden formar ondas estacionarias planas dentro de la cavidad resonante. Para el caso de cavidades resonantes sin paredes, como las del láser en cuestión, las soluciones estacionarias son estos modos TEM$_{pq}$. Los números pq son el orden de los polinomios de Hermite-Gauss presentes en la parte analítica de la solución.  %que son un resultado directo de la naturaleza cuantica del campo electromagnetico. 
(\textcolor{Blue}{Hay varios modos de entender esto, los que yo vi son analizar la relacion entre un haces paraxiales y las soluciones del oscilador cuantico, especialmente la relacion entre las autofunciones del nucleo de la transformada generalilzada de Fresnel para la matrix de transferencia ABCD que te aparece al analizar el sistema optico de la cavidad de manera "clasica" y las soluciones de un oscilador y la otra manera, mas sencilla, es darte cuenta que lo que estas viendo son los modos de oscilacion normales del campo electromagnetico, ahora, debido a que cuando cuantizas el campo electromagnetico, lo que tenes son osciladores cuanticos coherentes , cada modo TEM se corresponde con un oscilador cuantico coherente de tu campo electromagnetico cuantizado })
El modo más bajo es el 00, que presenta un perfil de intensidad gaussiano. Modos <<superiores>> son combinaciones de polinomios de Hermite-Gauss en los ejes $x$ e $y$, y los números pq indican la cantidad de ceros en el perfil de intensidad del campo de radiación electromagnética, en cada dirección[1].

Luego de visualizar los modos TEM, se buscó generar segunda armónica. Este es un fenómeno de óptica no lineal que consiste en producir un haz láser con la mitad de la longitud de onda original. Así, de un haz infrarrojo ($\lambda$ = 1064nm), se pudo obtener uno verde ($\lambda$ = 532nm)[2].


\section*{Desarrollo experimental}
La primera parte del experimento consistió en armar una cavidad estable y medir cómo afectaba la variación de corriente y de distancia a la potencia de salida del láser y del diodo de bombeo. Para configurar la cavidad, se utilizó un láser auxiliar de He--Ne (Melles Griot 05-LHR-111, $\lambda$ = 632.8 nm ), dos espejos de plata (New Focus 5151/vis) y un espejo plano con 98\% de reflectividad (el espejo que <<cierra>> la cavidad). Inicialmente se colocaron los espejos de Ag como muestra la Fig. \ref{cavplana}; el primero se encontraba a una distancia horizontal de $(148.5 \pm 0.1)$cm del láser de He--Ne, el segundo a $(169.5 \pm 0.1)$cm del medio óptico (también en la misma línea) y entre ambos espejos había una distancia diagonal de $(155.3 \pm 0.1)$cm. Los espejos estaban montados en posicionadores angulares que habilitaban todos los grados de libertad necesarios para alinearlos. Se comenzó alineando las reflexiones en los espejos de Ag. Una vez logrado, se colocó el espejo dieléctrico plano (R = 50cm, HR @ 1064nm, $\phi$1") a una distancia de $(4.9 \pm 0.1)$cm de la barrita YAG y se repitió el proceso de alineación. Luego, se alimentó el diodo láser con una corriente de 2.3A aproximadamente y se utilizó una tarjeta infrarroja (la longitud de onda central del láser, indicada por la hoja de datos, es de 1064nm) para verificar que el láser estuviera efectivamente funcionando. 

Con esta cavidad estable ya definida, se tomaron las siguientes mediciones:
\begin{itemize}
    \item Potencia de salida del láser en función de la corriente, para valores entre 0.5A y 2.35A. El barrido se hizo cada 5A entre 2A y 2.35A y cada 10A entre 0.5A y 2A. Distancia fija de $(11.5 \pm 0.1)$cm. 
    \item Potencia de salida del láser a distancias de $(11.5 \pm 0.1)$cm, $(19.0 \pm 0.1)$cm, $(25.0 \pm 0.1)$cm, $(29.1 \pm 0.1)$cm, $(35.5 \pm 0.1)$cm, $(39.7 \pm 0.1)$cm y $(44.2 \pm 0.1)$cm. Corriente fija de 2.39A. 
\end{itemize}
Para ello se utilizó un medidor de potencia óptica (Thorlabs, modelo S302C) y se colocó un obturador a la salida del láser para evitar que la luz propia del láser de bombeo interfiriera en las mediciones. Luego se removió el espejo plano (es decir, se desarmó el láser) y se midió potencia de bombeo del diodo láser en función de la corriente y en función de la distancia manteniendo los parámetros anteriores. 

La segunda parte de la experiencia consistió en registrar distintos modos TEM. Dado que con la cavidad plano--paralela habría resultado más complicado visualizarlos, se construyó una cavidad en V (Fig. \ref{cavV1}). Teniendo en cuenta las regiones de estabilidad (Fig. \ref{estabilidad}), se tomó a = $(38.4 \pm 0.1)$cm y b = $(39.7 \pm 0.1)$cm. Previamente se habían tomado a = $(51.0 \pm 0.1)$cm y b = $(16.0 \pm 0.1)$cm, pero con esas dimensiones no se logró lasear. Para armar la cavidad en V fue necesario, en primer lugar, alinear el láser con la plano--paralela. A continuación, se colocaron un espejo dieléctrico cóncavo  (R = 50cm, HR @ 1064nm, $\phi$1") en el extremo del brazo a y un espejo dieléctrico plano en el del brazo b. Una vez alineado el sistema nuevamente, se removió el espejo que formaba la cavidad plano--paralela (el que se encontraba a 4.9cm de la barrita YAG) y se comprobó que laseara.
Para tomar una fotografía del modo TEM$_{00}$ se colocó una hoja de papel cuadriculada detrás del espejo de salida y, detrás, una cámara web. Debido a la intensidad del haz, para obtener una fotografía no saturada, fue necesario disminuir la corriente de 2.39A a 1.57A. Para evitar este problema, se colocó un espejo plano de Ag para redireccionar el haz de salida hacia una pantalla colocada a unos 5m, aproximadamente. A esa distancia fue posible observar los modos TEM y tomarles una fotografía. Para obtener los distintos modos, se modificaron minimamente las dimensiones de la cavidad con los tornillos de los posicionadores angulares. 


\begin{figure}[H]
    \centering
    \includegraphics[scale=0.5]{Graficos/cavplana.png}
    \caption{Arreglo de la cavidad plano--paralela junto con el sistema de alineación compuesto por el láser de He--Ne y los espejos de Ag.}
    \label{cavplana}
\end{figure}


Por último, se generó segunda armónica. Para ello, se colocó, en el brazo b y cerca del espejo de salida, un cristal KTP (potassium titanyl phosphate). Luego, detrás del espejo de salida se ubicó un prisma para separar los haces verde e infrarrojo y se midió la variación de potencia de salida de cada haz en función de la corriente de bombeo, tal como se había hecho con el haz saliente de la cavidad plano--paralela.   







\begin{figure}[H]
    \centering
    \includegraphics[scale=0.5]{Graficos/cavV1.png}
    \caption{Arreglo de la cavidad en V. Detrás del espejo de salida se colocó un espejo de Ag para redireccionar el haz hacia una pantalla ubicada a, aproximadamente, 4m.}
    \label{cavV1}
\end{figure}

\begin{figure}[H]
    \centering
    \includegraphics[scale=0.27]{Graficos/estabilidad.png}
    \caption{Regiones de estabilidad para las cavidades en V. Se muestra un ejemplo de medidas posibles para ambos brazos.}
    \label{estabilidad}
\end{figure}


%------------------------------------------------

\section*{Resultados y análisis}
% \addcontentsline{toc}{section}{Resultados del Experimento} % Adds this section to the table of contents
Los resultados obtenidos en relación a la potencia óptica del láser fueron los esperados. En la Fig. \ref{laservscorr} se ve que la potencia óptica del láser es insignificante hasta que la corriente de bombeo alcanza 1.9A. A partir de ese umbral, la potencia aumenta de forma lineal  (R$^2$ = 0,99192, $\chi^2$ = 0,02477). % con pendiente $ (11,89 \pm 0,38)$mW/A, lo cual implica una ganancia de [\textcolor{cyan}{algo, si es que vale esta afirmación}].
\textcolor{cyan}{Resultaría interesante estudiar, en alguna experiencia futura, hasta qué valor de corriente se mantiene esta relación lineal}. 

La variación de la potencia óptica con la distancia (Fig. \ref{laservsdist}), a priori, parecen indicar que esta disminuye a medida que se aleja el sensor del láser. Sin embargo, teniendo en cuenta que una de las características de estos dispositivos es su gran colimación, se descarta esta suposición. Se cree que el porqué del comportamiento radica en la dirección del haz; es posible que este estuviera ligeramente desviado con respecto a la línea que une el láser con el sensor. Dado que el área activa del sensor es relativamente pequeña (12mm de diámetro) es posible que una pequeña desviación en el angulo horizontal del haz produjera estos errores. 
%Debe tenerse en cuenta, además, que debido a que el sensor estaba montado sobre un vástago, pudieron existir desviaciones angulares del cabezal del detector debidas a la orientación de este respecto al eje del vastago. 
Además, a pesar de haber utilizado un obturador, no se debe descartar que parte del haz del diodo de bombeo haya podido filtrarse y su potencia ser medida junto con la del láser. Esto podría explicar el comportamiento de los últimos tres puntos; de haber sido únicamente un problema de dirección del haz, los datos deberían haberse ubicado todos sobre una recta decreciente (como los primeros tres) hasta alcanzar una distancia en que la potencia fuera nula, pues el haz ya no estaría incidiendo sobre el sensor. 

Se decidió no incluir los resultados obtenidos en relación a la potencia del diodo de bombeo, dado que estas mediciones no fueron tomadas como correspondía y lo obtenido no explica el comportamiento real de la potencia. En una próxima experiencia, para medir la dependencia de la potencia óptica del diodo láser en función de la corriente y de la distancia, se deberá retirar la barra de Nd:YAG y colocar el sensor en el foco del sistema de enfoque (el haz de este láser tiene un perfil rectangular, y se utiliza un sistema de lentes que coliman el haz y lo enfocan en el Nd:YAG). Al medir a una distancia fija de $(11.5 \pm 0.1)$cm de la caja que protege el arreglo, se obtuvo un comportamiento no lineal de la potencia óptica, cuando se esperaba uno similar al de las Fig. \ref{laservscorr} y \ref{laservsdist}.

Se analizaron la forma de los perfiles de intensidad de los modos TEM a partir de las fotografias obtenidas. Primeramente se corrigieron los errores introducidos por la perspectiva al tomar las fotografías \textcolor{magenta}{[Cómo se corrigieron?]} (\textcolor{Blue}{No hay tanto problema en usar un programa de manipulacion de imagenes en tanto y en cuanto sepas lo que estas haciendo y porque lo estas haciendo, como dije en el mail, si conoces ALGUNA OTRA MEJOR MANERA de corregir los ERRORES INTRODUCIDOS por haber TOMADO LAS IMAGENES DESDE UN ANGULO INADECUADO , estoy ABIERTO A SUGERENCIAS}) y se aisló la región donde se observan los modos TEM propiamente dichos. (\textcolor{Blue}{poner algunas imagenes de los modos}) El procedimiento para analizar los perfiles de intensidad 
%de dichas imagenes 
consistió en cargar \textcolor{magenta}{[convertir, con el software Python u Origin, no sé que usaste]} (\textcolor{Blue}{ NO. el termino correcto es cargar, el archivo de imagen ALMACENA una matriz de datos, vos no 'convertis' lo que esta en el archivo de imagen en una matriz, de nuevo, el archivo ALMACENA una matriz})\textcolor{Green}{en Python con la libreria scikit-video } cada imagen a una matriz y sumar para cada columna los valores de intensidad de cada píxel, obteniéndose luego una suma acumulada de dicho vector, el cual posteriormente se normalizó \textcolor{Green}{para hacer comparables entre si los resultados de los diferentes perfiles} ( \textcolor{magenta}{[Por qué? Para qué? Finalidad?]}).(\textcolor{Blue}{poner algun grafico de la curva para los modos p y q de algun modo}). Esto se hizo para ambas direcciones longitudinales. Finalmente, se derivó la curva de intensidad para obtener, por separado, los perfiles correspondientes a cada dirección. [[[[Para comprobar que dichos perfiles se ajustan a los resultados teóricos se ajusto \textcolor{magenta}{(Los resultados experimentales no se ajustan a la teoría, la teoría se ajusta a los resultados!)[Para estudiar si los perfiles eran los esperados según el marco teórico con que se trabajó, se modelaron las curvas de dos formas diferentes.]}(\textcolor{Blue}{Justamente. Vos tenes un modelo teorico, vos queres saber si tus datos se ajustan a tu modelo teorico, para falsear o corroborar la hipotesis teorica sobre la que esta basada tu teoria. Vos identificas las fuentes de error, y tratas de eliminar las fuentes de error de tus datos antes de hacer una comparacion, si despues de eliminar errores y tratar de ajustar los datos experimentales a tu modelo teorico no hay manera , ninguna manera en que calcen, entonces descartar el modelo teorico }) estas curvas con el software Originlabs, ajustandose los perfiles de dos maneras diferentes.]]]]] \textcolor{magenta}{En primer lugar, se utilizó una función de Hermite--Gauss del orden correspondiente. Luego, a esa función se agregaron parámetros libres para compensar los errores generados por
\begin{itemize}
    \item el hecho de que el papel en donde se proyectó el haz láser no era totalmente negro,
    \item el fondo gaussiano producido por la radiacion del laser de bombeo, y
    \item la variacion tonal relativa debido a la reflexión del haz sobre el papel.
\end{itemize}
}

[[[[[En intento ajustar el perfil directamente con una funcion de Hermite-Gauss del orden esperado y en segundo lugar con una funcion a la que se le agrego a la funcion de Hermite-Gauss de orden apropiado parametros libres para compensar los errores introducidos por: a) el hecho de que el papel en donde se projecto el haz laser no era totalmente negro, b) el fondo gaussiano producido por la radiacion del laser de bombeo y c) la variacion tonal relativa debido a la reflexion del haz sobre el papel.]]]]]

Cuando se intento ajustar \textcolor{magenta}{[Del ajuste de]} las curvas sin compensar estos errores no se obtuvieron los resultados esperados (\textcolor{Blue}{ Poner el grafico de los modos TEM 22 'limpios' }). Sin embargo, para órdenes bajos de los modos TEM, el ajuste que \textcolor{magenta}{[contempló]} incluyó los errores introducidos por el procedimiento se correspondio con las curvas obtenidas \textcolor{magenta}{[No entiendo qué significa esto que escribiste. ¿Estás queriendo decir que el ajuste fue satisfactorio?]}(\textcolor{Blue}{Estoy queriendo decir que si tomas en cuenta las fuentes de error existen ajusten que convergen y se corroboran con los datos})(\textcolor{Blue}{ poner el grafico del ajuste 'compensado' del modo 22 }).(\textcolor{Blue}{Estoy queriendo decir que los errores de la medicion para el procedimiento que utilizamos dependen de que tan cerca esten los modos principalmente al error que te contribuye la reflexion de cada modo}) 
\textcolor{magenta}{Aun así, este modelo falló para órdenes mayores a 3 debido a que los máximos de intensidad estaban demasiado cerca, lo que impedía que fueran discernidos apropiadamente.}
[[[[[Aun así, el ajuste teniendo en cuenta los errores introducidos por el procedimiento experimental fallo tambien para ordenes mayores a 3 debido a que los maximos de intensidad estaban demasiado cerca como para ser discernidos apropiadamente. ]]]]

\iffalse
(\textcolor{Green}{ Bueno mas o menos te cuento como es el analisis de los modos TEM y vos despues lo reelaboras para que coincida con el estilo del informe. \newline En primer lugar de las fotografias obtenidas se aislo la region de la foto correspondiente al modo y se corrigio la perspectiva. Esto hay que hacerlo porque sino introduce errores a la hora de analizar el perfil. Obteniendose imagenes de los modos en escala de grises de 16-bits de profundidad. En segun lugar se analizo estas imagenes en Python con el modulo scikit-video, la idea es que para cada direccion de la imagen ( o sea primero analizas la matriz como viene y despues repetis para la transpuesta), o sea basicamente primero obtenes la curva de la integral en esa direccion , haciendo una suma de todas las intensidades de todos los pixeles en una linea y despues haciendo una "suma acumulada" de todos estos valores la cual normalize a 1 . Finalmente obtenes el perfil de intensidades derivando esta curva. Por ultimo lo que hice fue pasar todos estos datos a Origin y hacer un ajuste no lineal de los perfiles. Despues en las conclusiones hay que aclarar que no es posible hacer un ajuste directo de los modos gaussianos, debido a los errores o "ruidos" introducidos por la diferencia de contraste entre los modos, el fondo debido al laser de bombeo, el tono del papel y porque estamos tomando fotos de la imagen proyectada sobre el papel y no directamente sobre el sensor de la camara CCD . Si es posible ajustar para modos bajos si se compensan esos errores en el ajuste, pero para modos mayores a 3 es imposible porque los picos estan demasiado cerca y el ruido te tapa cualquier estructura fina.})\fi

Una vez generada segunda armónica y separados los haces verde e infrarrojo, se midió la potencia de cada uno y los resultados son los que se muestran en las Fig. \ref{potverde} y \ref{potrojo}. La potencia del haz infrarrojo mostró un comportamiento lineal, tal como se esperaba. En este caso, el umbral de corriente fue más bajo que para el caso de la Fig. \ref{laservscorr}; concretamente, se pasó de un umbral de 1.9A a uno de 1.46A. Si bien en ambos casos se estaba trabajando con cavidades estables, la plano--paralela es menos estable que la cavidad en V (por eso se utilizó esta configuración para observar los modos TEM), y por eso era necesaria más corriente para que funcionara el láser. Por cuestiones de tiempo no se registraron datos a corrientes menores de 1.46A, aunque sí se verificó, realizando un barrido de corriente, que la potencia para valores inferiores a 1.46A era nula. 

Con respecto a lo obtenido para la potencia del haz verde, no se tiene una explicación satisfactoria. Hay un punto (potencia a 1.1A) que presenta un salto. Dado que es un hecho aislado, se atribuye a un salto de tipo eléctrico y no relacionado con el comportamiento de la potencia en sí. Resultados de experiencias previas indicaban que este comportamiendo debería haber sido cuadrático a partir de un cierto umbral de corriente. Se tomó la región donde la potencia del haz infrarrojo era lineal y se modelaron los datos del haz verde con una función lineal y con una cuadrática. Como se puede ver, no hay una diferencia notable entre ambos modelos. Si bien los datos siguen un comportamiento ligeramente más parabólico, una recta, en este caso, también sería una aproximación adecuada. Por tanto, no se puede afirmar que haya un comportamiento cuadrático predominante. Una hipótesis es que puede haber cambiado el modo del láser con valores distintos de corrientes, aunque no se está en grado de corroborarlo. Se descarta una influencia del cristal KTP o del prisma, por ser materiales con alto umbral de daño y estabilidad térmica; asimismo, se tuvo la precaución de enfocar el sensor sobre cada haz por separado, centrándolo en el área activa. En estas condiciones, es necesario repetir las mediciones para poder explicar el comportamiento de la potencia del haz verde.


%[Igual me parece, al menos a mi, apropiado mencionar en el informe que para los niveles de potencia optica utilizados ( muy por debajo del umbral de daño ) y las caracteristicas del cristal y del prisma se ignoraron los errores producidos por inestabilidades termicas.]





\begin{figure}[H]
    \centering
    \includegraphics[scale=0.4]{Graficos/potvscor.png}
    \caption{Potencia óptica de láser en función de la corriente de bombeo. La potencia aumenta linealmente a partir de 1.9A.}
    \label{laservscorr}
\end{figure}

\begin{figure}[H]
    \centering
    \includegraphics[scale=0.4]{Graficos/potvsdist.png}
    \caption{Potencia óptica de láser en función de la distancia para una corriente de bombeo constante de 2.39A.}
    \label{laservsdist}
\end{figure}

\begin{figure}[H]
    \centering
    \includegraphics[scale=0.4]{Graficos/pot_infrarrojo.png}
    \caption{Potencia 2do armónico infrarrojo + ajuste lineal}
    \label{potrojo}
\end{figure}

\begin{figure}[H]
    \centering
    \includegraphics[scale=0.4]{Graficos/pot_verde1.png}
    \caption{Potencia 2do armónico verde}
    \label{potverde}
\end{figure}




\section{Conclusiones}
% \addcontentsline{toc}{section}{Conclusiones} % Adds this section to the table of contents

(\textcolor{Blue}{Escribo la parte de los modos TEM, decime si tengo que escribir algo mas})
Se pudieron observar de manera cualitativo que diferentes modos TEM obtenidos al variar la longitud de la cavidad en V se correspondendian con lo predicho por la teoria. Si bien en el analisis cuantitativo se tuvieron grandes dificultades al identificarse la correlacion con la teoria de manera directa debido a los errores introducidos por el procedimiento experimental, si se comprobo dicha correlacion al tenerse en cuenta dichos errores. Una de las mayores dificultades al obtener las imagenes de los modos es que al momento de la medicion no habia a nuestra disposicion camaras CCD en funcionamiento con las cuales obtener directamente informacion de los perfiles de intensidad.

%----------------------------------------------------------------------------------------
%	BIBLIOGRAPHY
%----------------------------------------------------------------------------------------


\begin{thebibliography}{100}
\bibitem{1}{Apuntes de la materia, \url{http://users.df.uba.ar/bragas/Labo5_1er2011/laser2k.pdf}}
\bibitem{2}{Robert W. Boyd, \textit{Nonlinear Optics}, 3ra edición, Academic Press, Inc, 2008.}
\end{thebibliography}
%----------------------------------------------------------------------------------------
\end{multicols}
\end{document}
