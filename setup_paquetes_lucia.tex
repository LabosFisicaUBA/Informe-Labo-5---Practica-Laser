\usepackage[sc]{mathpazo} % Use the Palatino font
\usepackage[T1]{fontenc} % Use 8-bit encoding that has 256 glyphs
\linespread{1.05} % Line spacing - Palatino needs more space between lines
%\usepackage{microtype} % Slightly tweak font spacing for aesthetics

\usepackage[twoside,width=16cm,height=24cm,left=3cm]{geometry}
%\usepackage[hmarginratio=1:1,top=20mm,width=19cm,height=23cm,columnsep=15pt]{geometry} % Document margins
\usepackage{multicol} % Used for the two-column layout of the document
\usepackage[hang, small,labelfont=bf,up,textfont=it,up]{caption} % Custom captions under/above floats in tables or figures
\usepackage{booktabs} % Horizontal rules in tables
\usepackage{float} % Required for tables and figures in the multi-column environment - they need to be placed in specific locations with the [H] (e.g. \begin{table}[H])
\usepackage{hyperref} % For hyperlinks in the PDF

%----------- Agregados para el caso de ustedes -------------------------------
\usepackage[spanish]{babel}% idioma castellano
\usepackage[utf8]{inputenc}% esto es para poder poner los tildes directamente. Puede que cambie de versión a versión de sistema operativos (más información en http://www.aq.upm.es/Departamentos/Fisica/agmartin/webpublico/latex/FAQ-CervanTeX/FAQ-CervanTeX-6.html )
\usepackage{graphicx} % para insertar figuras
\usepackage{subfigure} % para insertar figuras dentro de figuras
\usepackage{times} % plataforma
\usepackage{amsmath} % --para ecuaciones y algunos símbolos 
% ---------------------- -----------------------------------------------------

\usepackage{lettrine} % The lettrine is the first enlarged letter at the beginning of the text
%\usepackage{paralist} % Used for the compactitem environment which makes bullet points with less space between them


\usepackage{abstract} % Allows abstract customization
\renewcommand{\abstractnamefont}{\normalfont\bfseries} % Set the "Abstract" text to bold
\renewcommand{\abstracttextfont}{\normalfont\itshape} % Set the abstract itself to small italic text
\addto\captionsspanish{ % Modifica algunos nombres cambiandolos por los definidos a continuacion
        \def\contentsname{\'Indice}%
        \def\bibname{Referencias}%
        \def\tablename{Tabla}%
        \def\abstractname{Resumen}
        }
\usepackage[usenames,dvipsnames,svgnames,table]{xcolor}
%\usepackage{natbib}
%\usepackage[usenames]{color}
\usepackage{graphicx}
\usepackage[spanish]{babel}
\usepackage{amsmath}
\usepackage{float}
\usepackage{dsfont}
\usepackage{textcomp}
\usepackage{soul}
\usepackage{fancyhdr}
\usepackage{titlesec} % Allows customization of titles
\usepackage{fancyhdr} % Headers and footers
\usepackage[spanish]{babel}
\usepackage{amsmath}
%\usepackage{hyperref}


\pagestyle{fancy} % All pages have headers and footers
 \fancyhead{} % Blank out the default header
 \fancyfoot{} % Blank out the default footer
\fancyhead[C]{Laboratorio  $\bullet$ $\today$ } % Custom header text
\fancyfoot[RO,LE]{\thepage} % Custom footer text


