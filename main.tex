%%%%%%%%%%%%%%%%%%%%%%%%%%%%%%%%%%%%%%%%%
% Wenneker Article
% LaTeX Template
% Version 2.0 (28/2/17)
%
% This template was downloaded from:
% http://www.LaTeXTemplates.com
%
% Authors:
% Vel (vel@LaTeXTemplates.com)
% Frits Wenneker
%
% License:
% CC BY-NC-SA 3.0 (http://creativecommons.org/licenses/by-nc-sa/3.0/)
%
%%%%%%%%%%%%%%%%%%%%%%%%%%%%%%%%%%%%%%%%%

%----------------------------------------------------------------------------------------
%	PACKAGES AND OTHER DOCUMENT CONFIGURATIONS
%----------------------------------------------------------------------------------------


%\documentclass[10pt, a4paper, twocolumn]{article}
\documentclass[10pt, a4paper]{article}% 10pt font size (11 and 12 also possible), A4 paper (letterpaper for US letter) and two column layout (remove for one column)

\input{structure.tex} % Specifies the document structure and loads requires packages
\usepackage[sc]{mathpazo} % Use the Palatino font
\usepackage[T1]{fontenc} % Use 8-bit encoding that has 256 glyphs
\linespread{1.05} % Line spacing - Palatino needs more space between lines
%\usepackage{microtype} % Slightly tweak font spacing for aesthetics

\usepackage[twoside,width=16cm,height=24cm,left=3cm]{geometry}
%\usepackage[hmarginratio=1:1,top=20mm,width=19cm,height=23cm,columnsep=15pt]{geometry} % Document margins
\usepackage{multicol} % Used for the two-column layout of the document
\usepackage[hang, small,labelfont=bf,up,textfont=it,up]{caption} % Custom captions under/above floats in tables or figures
\usepackage{booktabs} % Horizontal rules in tables
\usepackage{float} % Required for tables and figures in the multi-column environment - they need to be placed in specific locations with the [H] (e.g. \begin{table}[H])
\usepackage{hyperref} % For hyperlinks in the PDF

%----------- Agregados para el caso de ustedes -------------------------------
\usepackage[spanish]{babel}% idioma castellano
\usepackage[utf8]{inputenc}% esto es para poder poner los tildes directamente. Puede que cambie de versión a versión de sistema operativos (más información en http://www.aq.upm.es/Departamentos/Fisica/agmartin/webpublico/latex/FAQ-CervanTeX/FAQ-CervanTeX-6.html )
\usepackage{graphicx} % para insertar figuras
\usepackage{subfigure} % para insertar figuras dentro de figuras
\usepackage{times} % plataforma
\usepackage{amsmath} % --para ecuaciones y algunos símbolos 
% ---------------------- -----------------------------------------------------

\usepackage{lettrine} % The lettrine is the first enlarged letter at the beginning of the text
%\usepackage{paralist} % Used for the compactitem environment which makes bullet points with less space between them


\usepackage{abstract} % Allows abstract customization
\renewcommand{\abstractnamefont}{\normalfont\bfseries} % Set the "Abstract" text to bold
\renewcommand{\abstracttextfont}{\normalfont\itshape} % Set the abstract itself to small italic text
\addto\captionsspanish{ % Modifica algunos nombres cambiandolos por los definidos a continuacion
        \def\contentsname{\'Indice}%
        \def\bibname{Referencias}%
        \def\tablename{Tabla}%
        \def\abstractname{Resumen}
        }
\usepackage[usenames,dvipsnames,svgnames,table]{xcolor}
%\usepackage{natbib}
%\usepackage[usenames]{color}
\usepackage{graphicx}
\usepackage[spanish]{babel}
\usepackage{amsmath}
\usepackage{float}
\usepackage{dsfont}
\usepackage{textcomp}
\usepackage{soul}
\usepackage{fancyhdr}
\usepackage{titlesec} % Allows customization of titles
\usepackage{fancyhdr} % Headers and footers
\usepackage[spanish]{babel}
\usepackage{amsmath}
%\usepackage{hyperref}


\pagestyle{fancy} % All pages have headers and footers
 \fancyhead{} % Blank out the default header
 \fancyfoot{} % Blank out the default footer
\fancyhead[C]{Laboratorio  $\bullet$ $\today$ } % Custom header text
\fancyfoot[RO,LE]{\thepage} % Custom footer text


 %incluye los paquetes usados mas comunes 

%----------------------------------------------------------------------------------------
%	INFORMACION DEL ARTICULO
%----------------------------------------------------------------------------------------

\title{Estudio y caracterización de  modos transversales electromagnéticos y cavidades de oscliacion de un laser Nd:YAG } % Titulo del Informe

\author{
	\authorstyle{Lucia Evangelista Gallo \textsuperscript{1,1}}
	\authorstyle{Leandro Ariel Pezzente\textsuperscript{1,1}} % Authors
	\newline\newline % Space before institutions
	\textsuperscript{1}\institution{Facultad de Ciencias Exactas y Naturales}\\ % Institution 1
	\textsuperscript{1}\institution{Universidad Nacional de Buenos Aires, Buenos Aires, Argentina}\\ % Institution 1
	\keywordname{Laser --- Nd:YAG --- Modos Transversales --- Cavidades de Oscilación } % Keywords - if you don't want any simply remove all the text between the curly brackets
}

\date{} % Add a date here if you would like one to appear underneath the title block, use \today for the current date, leave empty for no date

%----------------------------------------------------------------------------------------

\begin{document}
\maketitle % Print the title
\thispagestyle{fancy} % All pages have headers and footers
%\thispagestyle{firstpage} % Apply the page style for the first page (no headers and footers)

%----------------------------------------------------------------------------------------
%	ABSTRACT
%----------------------------------------------------------------------------------------

\begin{abstract}
%\lettrineabstract{ 
En este experimento se busca estudiar el comportamiento de la cavidad resonante de un láser de Nd:YAG. Se procede primero a determinar las condiciones de alineación bajo las cuales un espejo dieléctrico de 98\% permite la aparición de modos resonantes en una cavidad lineal , así como también se estudian las características de la potencia óptica emitida por el láser de Nd:YAG tanto en cavidades lineales como en cavidades en V. Finalmente, se analizan los distribución espacial de los diferentes modos transversales electromagnéticos emitidos por el laser  
%}
\end{abstract}
\begin{multicols}{2} % Two-column layout throughout the main article text
%----------------------------------------------------------------------------------------
%	ARTICLE CONTENTS
%----------------------------------------------------------------------------------------
\tableofcontents % Print the contents section

\section{Introducción}
% \addcontentsline{toc}{section}{Introduccion} % Adds this section to the table of contents

\subsection{Estructura y composicion de un laser}

El láser es un dispositivo que permite la amplificación de radiación electromagnética mediante un proceso físico conocido como emisión estimulada o inducida. Este proceso permite la amplificación de señales lumínicas generadas por otros medios. El mecanismo que genera dichas señales lumínicas se denominada mecanismo de bombeo, y los procesos para generarlo son diversos y pueden ir desde descargas eléctricas en medios gaseoso hasta el uso de otros laseres en la excitación fluorescente de medios orgánicos como colorantes. Esto es necesario para excitar las transiciones electrónicas en el medio en el que se produce la emisión estimulada, denominado medio activo,debido a que se debe entregar energía al medio activo de manera constante para que pueda mantenerse la emisión estimulada. Por ultimo, se necesita de un mecanismo de realimentación a fin de inducir las transiciones electrónicas que producen la amplificación de la radiación electromagnética. El mecanismo de retroalimentación se logra mediante una cavidad resonante que usualmente consiste en dos o mas espejos de alta reflectividad alineados de tal manera de se favorezca la la reinyeccion de la luz emitida en el sentido del eje de la cavidad por el medio activo excitado al mismo medio. Los espejos entonces actúan como un multiplicador de la longitud de onda del material. El mecanismo de retroalimentacion es indispensable porque para que la el láser emita, el proceso de bombeo no debe solamente excitar el medio activo, sino que que también debe lograr la condición de población de inversión en la cual hay mas electrones del medio activo en estados excitados de alta energía que en estados de baja energía.
Otro aspecto importante por el cual es crucial la correcta alineación de los espejos en la cavidad, es que , puesto que como el resonador debe sustentar ondas estacionarias de luz, es decir, se debe mantener una configuración estable del campo de radiación.
\newline
Las características mas distintivas de los laseres son que la luz emitida por estos es cuasi-monocromática, coherente y altamente colimada.

\subsection{Condiciones de estabilidad de una cavidad laser }
Para un resonador optico dado en un laser se tienen modos de oscilacion que dependen son o bien dependientes de la frecuencia del haz o bien dependientes de la distribucion espacial del haz. Estos ultimos modos se denominan modos transversales electromagneticos, y son una consecuencia del hecho de el haz de luz no es una onda plana sino que posee una distribucion espacial finita. Por lo general, un haz laser emitido por una cavidad resonante es una combinacion de varios modos de oscilacion. Esto es importante al analizar las perdidas por difraccion las cuales son mas altas para modos transversales de mayor orden y afectan mas seriamente a la calidad del haz.

\section{Desarrollo Experimental}
% \addcontentsline{toc}{section}{Desarrollo Experimental} % Adds this section to the table of contents

 lobortis pellentesque.


\begin{align}
	A = 
	\begin{bmatrix}
		A_{11} & A_{21} \\
		A_{21} & A_{22}
	\end{bmatrix}
\end{align}




%------------------------------------------------


%------------------------------------------------

\subsection{Subsection}

In hac habitasse platea dictumst. Etiam ac tortor fermentum, ultrices libero gravida, blandit metus. Vivamus sed convallis felis. Cras vel tortor sollicitudin, vestibulum nisi at, pretium justo. Curabitur placerat elit nunc, sed luctus ipsum auctor a. Nulla feugiat quam venenatis nulla imperdiet vulputate non faucibus lorem. Curabitur mollis diam non leo ullamcorper lacinia.

% Morbi iaculis posuere arcu, ut scelerisque sem. Class aptent taciti sociosqu ad litora torquent per conubia nostra, per inceptos himenaeos. Mauris placerat urna id enim aliquet, non consequat leo imperdiet. Phasellus at nibh ut tortor hendrerit accumsan. Phasellus sollicitudin luctus sapien, feugiat facilisis risus consectetur eleifend. In quis luctus turpis. Nulla sed tellus libero. Pellentesque metus tortor, convallis at tellus quis, accumsan faucibus nulla. Fusce auctor eleifend volutpat. Maecenas vel faucibus enim. Donec venenatis congue congue. Integer sit amet quam ac est aliquam aliquet. Ut commodo justo sit amet convallis scelerisque.

\begin{enumerate}
	\item First numbered item in a list
	\item Second numbered item in a list
	\item Third numbered item in a list
\end{enumerate}

Aliquam elementum nulla at arcu finibus aliquet. Praesent congue ultrices nisl pretium posuere. Nunc vel nulla hendrerit, ultrices justo ut, ultrices sapien. Duis ut arcu at nunc pellentesque consectetur. Vestibulum eget nisl porta, ultricies orci eget, efficitur tellus. Maecenas rhoncus purus vel mauris tincidunt, et euismod nibh viverra. Mauris ultrices tellus quis ante lobortis gravida. Duis vulputate viverra erat, eu sollicitudin dui. Proin a iaculis massa. Nam at turpis in sem malesuada rhoncus. Aenean tempor risus dui, et ultrices nulla rutrum ut. Nam commodo fermentum purus, eget mattis odio fringilla at. Etiam congue et ipsum sed feugiat. Morbi euismod ut purus et tempus. Etiam est ligula, aliquam eget porttitor ut, auctor in risus. Curabitur at urna id dui lobortis pellentesque.

\begin{table}
	\caption{Example table}
	\centering
	\begin{tabular}{llr}
		\toprule
		\multicolumn{2}{c}{Name} \\
		\cmidrule(r){1-2}
		First Name & Last Name & Grade \\
		\midrule
		John & Doe & $7.5$ \\
		Richard & Miles & $5$ \\
		\bottomrule
	\end{tabular}
\end{table}

%------------------------------------------------

\section{Resultados del Experimento}
% \addcontentsline{toc}{section}{Resultados del Experimento} % Adds this section to the table of contents

\section{Analisis y Conclusiones}
% \addcontentsline{toc}{section}{Analisis y Conclusiones} % Adds this section to the table of contents

\begin{figure}
	\includegraphics[width=\linewidth]{bear.jpg} % Figure image
	\caption{A majestic grizzly bear} % Figure caption
	\label{bear} % Label for referencing with \ref{bear}
\end{figure}

In hac habitasse platea dictumst. Vivamus eu finibus leo. Donec malesuada dui non sagittis auctor. Aenean condimentum eros metus. Nunc tempus id velit ut tempus. Quisque fermentum, nisl sit amet consectetur ornare, nunc leo luctus leo, vitae mattis odio augue id libero. Mauris quis lectus at ante scelerisque sollicitudin in eu nisi. Nulla elit lacus, ultricies eu erat congue, venenatis semper turpis. Ut nec venenatis velit. Mauris lacinia diam diam, ac egestas neque sodales sed. Curabitur eu diam nulla. Duis nec turpis finibus, commodo diam sed, bibendum erat. Nunc in velit ullamcorper, posuere libero a, mollis mauris. Nulla vehicula quam id tortor ornare blandit. Aenean maximus tempor orci ultrices placerat. Aenean condimentum magna vulputate erat mattis feugiat.

Quisque lacinia, purus id mattis gravida, sem enim fringilla erat, non dapibus est tellus pellentesque velit. Vivamus pretium sem quis leo placerat, at dignissim ex iaculis. Donec neque tortor, pharetra quis vestibulum id, tempus scelerisque mi. Cras in mattis est. Integer nec lorem rutrum, semper ligula bibendum, iaculis neque. Sed in nunc placerat, viverra dui in, fringilla sem. Sed quis rutrum magna, vitae pellentesque eros.

% Praesent maximus mauris vitae nisl pulvinar, at tristique tortor aliquam. Etiam sit amet nunc in nulla vulputate sollicitudin. Aliquam erat volutpat. Praesent pharetra gravida cursus. Quisque vulputate lacus nunc. Integer orci ex, porttitor quis sapien id, eleifend gravida mi. Etiam efficitur justo eget nulla congue mattis. Duis commodo vel arcu a pretium. Aenean eleifend viverra nisl, nec ornare lacus rutrum in.

% Vivamus pulvinar ac eros eu pellentesque. Duis nibh felis, sagittis sed lacus at, sagittis mattis nisi. Fusce ante dui, tincidunt in scelerisque ut, sagittis at magna. Fusce tincidunt felis et odio tincidunt imperdiet. Cras ut facilisis nisl. Aliquam vitae consequat metus, eget gravida augue. In imperdiet justo quis nulla venenatis accumsan. Aliquam aliquet consectetur tortor, at sollicitudin sapien porta sed. Donec efficitur mauris id rhoncus volutpat. Vestibulum ante ipsum primis in faucibus orci luctus et ultrices posuere cubilia Curae; Sed bibendum purus dapibus tincidunt euismod. Nullam malesuada ultrices lacus, ut tincidunt dolor. Etiam imperdiet quam eget elit tincidunt scelerisque. Curabitur ut ullamcorper dui. Cras gravida porta leo, ut lobortis nisl venenatis pulvinar. Proin non semper nulla.

% Praesent pretium nisl purus, id mollis nibh efficitur sed. Sed sit amet urna leo. Nulla sed imperdiet sem. Donec ut diam tristique, faucibus ligula vel, varius est. In ipsum ligula, elementum vitae velit ac, viverra tincidunt enim. Phasellus gravida diam id nisl interdum maximus. Ut semper, tortor vitae congue pharetra, justo odio commodo urna, vel tempus libero ex et risus. Vivamus commodo felis non venenatis rutrum. Sed pulvinar scelerisque augue in porta. Sed maximus libero nec tellus malesuada elementum. Proin non augue posuere, pellentesque felis viverra, varius urna. Lorem ipsum dolor sit amet, consectetur adipiscing elit. Donec dignissim urna eget diam dictum, eget facilisis libero pulvinar.

% Aliquam ex tellus, hendrerit sed odio sit amet, facilisis elementum enim. Suspendisse potenti. Integer molestie ac augue sit amet fermentum. Vivamus ultrices ante nulla, vitae venenatis ipsum ullamcorper sed. Phasellus gravida felis sapien, ac porta purus pharetra quis. Sed eget augue tellus. Nam vitae hendrerit arcu, id iaculis ipsum. Pellentesque sed magna tortor.

% In ac tempus diam. Sed nec lobortis massa, suscipit accumsan justo. Quisque porttitor, ligula a semper euismod, urna diam dictum sem, sed maximus risus purus sit amet felis. Fusce elementum maximus nisi a mattis. Nulla vitae elit erat. Integer sit amet commodo risus, eget elementum nulla. Donec ultricies erat sit amet sem commodo iaculis. Donec euismod volutpat lacus, ut tempor est lacinia a. Vivamus auctor condimentum tincidunt. Praesent sed finibus urna. Sed pellentesque blandit magna et rhoncus.

% Integer vel turpis nec tellus sodales malesuada a vel odio. Fusce et lectus eu nibh rhoncus tempus vel nec elit. Suspendisse commodo orci velit, lacinia dictum odio accumsan et. Vivamus libero dui, elementum vel nibh non, fermentum venenatis risus. Aliquam sed sapien ac orci sodales tempus a eget dui. Morbi non dictum tortor, quis tincidunt nibh. Proin ut tincidunt odio.

% Pellentesque ac nisi dolor. Pellentesque maximus est arcu, eu scelerisque est rutrum vitae. Mauris ullamcorper vulputate vehicula. Praesent fermentum leo ac velit accumsan consectetur. Aliquam eleifend ex eros, ut lacinia tellus volutpat non. Pellentesque sit amet cursus diam. Maecenas elementum mattis est, in tincidunt ex pretium ac. Integer ultrices nunc rutrum, pretium sapien vitae, lobortis velit.

%----------------------------------------------------------------------------------------
%	BIBLIOGRAPHY
%----------------------------------------------------------------------------------------

\printbibliography[title={Bibliography}] % Print the bibliography, section title in curly brackets

%----------------------------------------------------------------------------------------
\end{multicols}
\end{document}
